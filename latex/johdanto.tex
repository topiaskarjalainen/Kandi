\chapter{Johdanto}\label{johd}

Tilastotieteissä \textit{frekventistinen} koulukunta oli pitkään vallitseva koulukunta. Viimeaikoina kuitenkin suosiotaan on kasvattanut \emph{bayesilainen} koulukunta. Aiemmin bayesilainen päättely ei päässyt leviämään, sillä toisin kuin frekventistinen koulukunta, bayesilainen ajattelu ei tarjonnut suurinpaan osaan kysymyksiä analyyttisiä ratkaisuja. Vasta tietokoneiden aikakautena \textit{Markovin ketju Monte Carlo -menetelmät} (MCMC-menetelmät) ovat antaneet mahdollisuuden ratkaista epä-triviaaleja ongelmia bayesilaisessa kehyksessä.

\emph{Monte Carlo -menetelmän} kehitteli 50-luvulla \textit{Los Alamosissa} työskennelleet \textit{Nicholas Metropolis}, \textit{Stanislav Ulam} ja ehkä 1900-luvun merkityksellisin tiedemies \textit{John von Neumann}. Yleinen määritelmä Monte Carlo -menetelmälle on toistuva satunnainen arvojen arpominen, aivan kuten \emph{Monte Carlon} kasinopöydissä toistuvasti arvotaan satunnaisesti uusia arvoja korttipakasta, rulettipyörästä ja nopista. Nimensä menetelmä onkin saanut juuri tästä. Yksinkertainen esimerkki Monte Carlo simuloinnista on esimerkiksi $\pi$:n arvon estimointi arpomalla sattumanvaraisesti pisteitä tasosta, ja laskemalla kuinka moni niistä on ympyrän säteen sisällä. 

\textit{Markovin ketjut} taas ovat \textit{stokastisia prosesseja}, jotka on nimetty venäläisen matemaatikon \textit{Andrey Markovin} mukaan. Niitä karakterisoi \emph{Markovin ominaiusuus}, jota avataan kappaleessa \ref{kappale: mc}.

Tässä tutkielmassa tulen ensin antamaan lyhyen johdatuksen Markovin ketjuihin ja selostan MCMC-menetelmien kannalta relevantin teorian. Tulen esittelemään lyhyesti kaksi algoritmia, joita käytetään MCMC-menetelmissä, \textit{Gibbsin otanta-algoritmin} ja \textit{Metropolis–Hastingsin algoritmin}. Esitän myös kaksi tärkeää diagnostiikkaa menetelmien tulosten arviointiin ja lopuksi vielä tarkastellaan käytännön esimerkkiä MCMC-algoritmistä.
