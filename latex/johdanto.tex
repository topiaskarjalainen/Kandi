\chapter{Johdanto}\label{johd}

Tilastotieteiss� \textit{frekventistinen} koulukunta oli pitk��n vallitseva koulukunta. Bayesil�inen p��ttely ei p��ssyt levi�m��n, sill� toisin kuin frekventistinen koulukunta, Bayesil�isyys ei tarjonnut suurinpaan osaan kysymyksi� analyyttisi� ratkaisuja. Vasta tietokoneiden aikakautena \textit{Markovin ketju Monte Carlo -menetelm�t} (MCMC-menetelm�t) ovat antaneet mahdollisuuden ratkaista \textit{posteriori}-jakaumat monimutkaisemmilta malleilta.

Monte Carlo menetelm�n kehitteli 50-luvulla \textit{Los Alamosissa} ty�skennelleet \textit{Nicholas Metropolis}, \textit{Stanislav Ulam} ja yleisnero \textit{John von Neumann}. 

T�ss� kanditutkielmassa aion selvent�� MCMC-menetelmien teoriaa, sek� esitt�� yleisimm�t kaksi algoritmia: \textit{Gibbs}- ja \textit{Metropolis--Hastings} algoritmit.