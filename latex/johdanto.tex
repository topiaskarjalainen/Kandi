\chapter{Johdanto}\label{johd}

Tilastotieteiss� \textit{frekventistinen} koulukunta oli pitk��n vallitseva koulukunta. Viimeaikoina kuitenkin suosiotaan on kasvattanut Bayesilainen koulukunta. Aiemmin Bayesil�inen p��ttely ei p��ssyt levi�m��n, sill� toisin kuin frekventistinen koulukunta, Bayesil�isyys ei tarjonnut suurinpaan osaan kysymyksi� analyyttisi� ratkaisuja. Vasta tietokoneiden aikakautena \textit{Markovin ketju Monte Carlo -menetelm�t} (MCMC-menetelm�t) ovat antaneet mahdollisuuden ratkaista \textit{posteriori}-jakaumat monimutkaisemmilta malleilta.

Monte Carlo menetelm�n kehitteli 50-luvulla \textit{Los Alamosissa} ty�skennelleet \textit{Nicholas Metropolis}, \textit{Stanislav Ulam} ja yleisnero \textit{John von Neumann}. Yleinen m��ritelm� Monte Carlo menetelm�lle on toistuva satunnainen arvojen arpominen. Yksinkertainen esimerkki Monte Carlo simuloinnista on esimerkiksi $\pi$:n arvon estimointi arpomalla sattumanvaraisesti pisteit� tasosta, ja laskemalla kuinka moni niist� on ympyr�n s�teen sis�ll�. 

\textit{Markovin ketjut} ovat \textit{stokastisia prosesseja}, jotka on nimetty ven�l�isen matemaatikon \textit{Andrey Markov}'n mukaan. 

MCMC-menetelmi� k�ytet��n nyky��n enimm�kseen tilastotieteess� laaja-alaisesti perinteisest� parametriestimoinnista aina vaalitulosten ennustukseen. Niill� on my�s sovelluksia biologiassa, fysiikassa ja kielitieteiss�.

