\chapter{Johdanto}\label{johd}

Tilastotieteissä \textit{frekventistinen} koulukunta oli pitkään vallitseva koulukunta. Viimeaikoina kuitenkin suosiotaan on kasvattanut Bayesilainen koulukunta. Aiemmin Bayesiläinen päättely ei päässyt leviämään, sillä toisin kuin frekventistinen koulukunta, Bayesiläisyys ei tarjonnut suurinpaan osaan kysymyksiä analyyttisiä ratkaisuja. Vasta tietokoneiden aikakautena \textit{Markovin ketju Monte Carlo -menetelmät} (MCMC-menetelmät) ovat antaneet mahdollisuuden ratkaista \textit{posteriori}-jakaumat monimutkaisemmilta malleilta.

Monte Carlo menetelmän kehitteli 50-luvulla \textit{Los Alamosissa} työskennelleet \textit{Nicholas Metropolis}, \textit{Stanislav Ulam} ja yleisnero \textit{John von Neumann}. Yleinen määritelmä Monte Carlo menetelmälle on toistuva satunnainen arvojen arpominen. Yksinkertainen esimerkki Monte Carlo simuloinnista on esimerkiksi $\pi$:n arvon estimointi arpomalla sattumanvaraisesti pisteitä tasosta, ja laskemalla kuinka moni niistä on ympyrän säteen sisällä. 

\textit{Markovin ketjut} ovat \textit{stokastisia prosesseja}, jotka on nimetty venäläisen matemaatikon \textit{Andrey Markovin} mukaan. 

Tässä tutkielmassa tulen ensin antamaan lyhyen johdatuksen Markovin ketjuista ja selostan MCMC-menetelmien kannalta relevantin teorian. Tulen esittelemään lyhyesti kaksi algoritmia, joita käytetään MCMC-menetelmissä, \textit{Gibbsin otanta-algoritmin} ja \textit{Metropolis–Hastings algoritmin}. Esitän myös kaksi tärkeää diagnostiikkaa menetelmien tulosten arviointiin ja lopuksi vielä tarkastellaan hieman laajempaa käytännön esimerkkiä MCMC-algoritmistä.
