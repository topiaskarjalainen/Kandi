


\chapter{Teoriaa}\label{teor}

\section{Perusmääritelmiä}

Määritellään ensiksi todennäköisyys.

\begin{maar}
	\textit{$\sigma$-algebra.} Olkoot $\Omega$ mielivaltainen epätyhjä joukko. Sigma-algebra perusjoukolla $\Omega$ on sen osajoukkojen joukkoperhe $\mathcal{F}$, joka toteuttaa ehdot:
	
	\begin{enumerate}
		\item $\emptyset\in\mathcal{F}$
		\item jos $A\in\mathcal{F},\ niin\ A^c \in\mathcal{F}$
		\item jos jos $A_k\in\mathcal{F},\ kaikilla\ k\in K$, missä $K$ on numeroituva joukko, niin $\bigcup_{k\in K} A_k \in \mathcal{F}$
	\end{enumerate}
\end{maar}

\begin{maar}
	Kuvaus \pr\ liittää kuhunkin tapahtumaan $A$ \textit{todennäköisyyden}, joka on luku suljetulla välillä [0,1] ja sille pätee:
	\begin{enumerate}
		\item $\pr(\Omega)=1$
		\item Jos $A$ on tapahtuma, niin sen komplementtitapahtuman $A^c$ todennäköisyys on $\pr(A^c)= 1 - \pr(A)$
		\item Jos $(A_k)_{k\in\N}$ ovat erillisiä tapahtumia, niin 
		\begin{displaymath}
			\pr(\bigcup_{k\in\N} A_k) = \sum_{k\in\N}\pr(A_k)
		\end{displaymath}
	\end{enumerate}
\end{maar}

\begin{maar}
	Olkoot $\mathcal{A}$ $\sigma$-algebra, ja olkoot $X$ joukko. Pari $(X, \mathcal{A})$ on mitallinen avaruus.
\end{maar}

\begin{maar}
	Kolmikkoa $(\Omega, \mathcal{F}, \pr )$ kutsutaan \textit{todennäköisyysavaruudeksi}.
\end{maar}

\begin{maar}
	\textit{Satunnaismuuttuja} $X$ on (lähes) mielivaltainen kuvaus $X:\Omega\rightarrow S$, jossa $S$ on \textit{tilajoukko}. 
\end{maar}

%%%%%%%%%%%%%%%%%%%%%%%%%%%%%%%%%
%%  Tässä Markovin ketjuista   %%
%%%%%%%%%%%%%%%%%%%%%%%%%%%%%%%%%

\section{Markovin ketjut}

Esitellään ensiksi joitain perus asioita Markovin Ketjuista, sillä ne eivät kuulu sellaisenaan opetussuunnitelmaan. Aloitetaan määrittelemällä stokastinen prosessi.

\begin{maar}
	Jono $(X_n:n=1,2,3,...)$ satunnaismuuttujia on diskreettiaikainen \textit{stokastinen prosessi}.
\end{maar}

\begin{merk}
	Merkitään stokastista prosessia merkinnällä $( X_n )$
\end{merk}

Määritellään nyt siirtymäydin, eli jakauma, joka määrittelee tilojen välisten siirtymien todennäköisyydet ja esitetään sitten \textit{Markovin ehto} diskreetille tila-avaruudelle määritelmässä \ref{markovin-ehto-d}, ja laajennetaan se jatkuvalle tila-avaruudelle määritelmässä \ref{markov-maar-c}. \cite{monte_carlo_book}

\begin{maar}
	Olkoot $(S,\mathcal{S})$ ja $(T,\mathcal{T})$ mitallisia avaruuksia. Siirtymäydin on funktio $T:S\times \mathcal{T} \rightarrow [0,\infty[$, jolle pätee
	\begin{enumerate}[(i)]
		\item $\forall s \in S: A \rightarrow T(s,A)$ on todennäköisyysmitta.
		\item $\forall A \in \mathcal{T}: s \rightarrow T(s,A)$ on mitallinen
	\end{enumerate}
	Diskreetissä tapauksessa siirtymäydintä kutsutaan \textit{siirtymä matriisiksi}, joka on 
	\begin{equation}
		p_{ij} = \pr(X_n = i | X_{n-1} = j), \:\forall i,j \in S
	\end{equation}
	Jatkuvassa tapauksessaydin kuvaa sitä ehdollista todennäköisyyttä, että siirtymä tapahtuu, eli $P(X \in A|x) = \int_A T(x,x')dx'$. Tälle pätee, että 
	\begin{equation}\label{siirt-tiheys}
		\int_S T(x,x')dx' = 1
	\end{equation}
\end{maar}

\begin{maar}\label{markovin-ehto-d}
	Stokastinen prosessi $(X_n)$ on \textit{Markovin ketju}diskreetissä tila-avaruudessa, jos kaikilla alkuhetkillä $m,n$ ja tiloilla $i,j\in S$ on voimassa
	\begin{equation}\label{markov-property}
		\begin{split}
			&\pr(X_{n+1}=j|X_0=i_0,X_1=i_1,...,X_{n-1}=i_{n-1},X_n=i) \\
		 &= \pr(X_{n+1}=j|X_n=i) 
		\end{split}
	\end{equation}
	ja \textit{siirtymätodennäköisyyksille} on voimassa 
	\begin{equation}\label{stationary}
		p_{ij}=\pr(X_{n+1}=j|X_n=i)=\pr(X_{m+1}=j|X_m=i)
	\end{equation}
	Yhtälöä \ref{markov-property} kutsutaan \textit{Markovin-ehdoksi} ja yhtälöä \ref{stationary} taas kutsutaan \textit{stationarisuusehdoksi}, mikä tarkoittaa, 
	että siirtymätodennäköisyys tilojen $i\ \text{ja}\ j$ välillä ei riipu ajasta $m\ \text{ja}\ n$, vaan pelkästään tiloista $i\ \text{ja}\ j$.
\end{maar}

\begin{maar}\label{markov-maar-c}
	Stokastinen prosessi $(X_n)$ on \textit{Markovin ketju} jatkuvassa tila-avaruudessa, jos kaikilla alkuhetkillä $t$, $X_t$:n ehdolliselle jakaumalle pätee
	\begin{equation}
		P(X_t\in A | x_{t-1},x_{t-2},...,x_0) = 	P(X_t\in A | x_{t-1})
	\end{equation}
\end{maar}

\begin{maar}
	Satunnaismuuttujan $X_0$ jakaumaa kutsutaan \textit{alkujakaumaksi}. 
\end{maar}

MCMC-menetelmien kannalta keskeinen ominaisuus Markovin ketjulle on sen tasapaino jakauma (eng. \textit{invariant distribution}). Siihen perustuu koko idea menetelmän takana. Esitetään seuraavaksi tämä ominaisuus, sekä määritellään toinen ominaisuus eli \textit{kääntyvä Markovin ketju} esittämällä ns. \textit{detailed balance}-yhtälö, jota tarvitaan Metropolis-hastings algoritmin tasapainojakauman olemassa olon osoittamiseen. 

\begin{maar}
	Todennäköisyysjakauma $\pi=(\pi)_{i\in S}$ on diskreetin tila-avaruuden Markovin ketjun $(X_n)$ tasapainojakauma, jos 
	\begin{equation}\label{stationary-dist}
		\sum_{i\in S} \pi_i p_{ij}=\pi_j, \forall j\in S
	\end{equation}
	Yhtälö \ref{stationary-dist} voidaan kirjoittaa myös muotoon 
	\begin{equation}\label{stationary-dist2}
		\pi^T\pr= \pi^T
	\end{equation}
	
	Jakauma $\pi$ on jatkuvan tila-avaruuden Markovin ketjun $( X_n )$ tasapainojakauma jos 
	\begin{equation}
		\pi(y) = \int_S \pi(x) T(x,y) dx
	\end{equation}
\end{maar}

\begin{maar}\label{kaant-disk}
	Markovin ketju on \textit{kääntyvä}, jos löytyy sellainen TN-jakauma $\lambda=(\lambda_i)_{i\in S}$, että 
	\begin{equation}
		\lambda_ip_{ij}= \lambda_jp_{ji},\forall i,j\in S
	\end{equation}
\end{maar}

\begin{maar}\label{kaant-jatk}
	Markovin ketju jatkuvassa $S$:ssä on kääntyvä, jos on olemassa 
	\begin{equation}
		\pi(x)T(x,y)=\pi(y)T(y,x), \forall x,y\in S
	\end{equation}
\end{maar}

Kääntyvällä Markovin ketjulla on sellainen mukava ominaisuus, että ketjun kääntyvyys on riittävä ehto tasapainojakauman olemassa ololle. Osoitamme tämän seuraavaksi.

\begin{lause}
	Jos Markovin ketju on kääntyvä, niin $\lambda=\pi$ on sen tasapainojakauma.
\end{lause}
\begin{proof}
	\begin{equation*}
		\sum_{i\in S} \lambda_i p_{ij} = \sum_{i\in S} \lambda_j p_{ji} = \lambda_j \sum_{i\in S} p_{ij} = \lambda_j
	\end{equation*}
\end{proof}

\begin{lause}
	Jos Markovin ketju $(X_n)$ on kääntyvä ja tilajoukko $S$ on jatkuva, niin $\pi$ on sen tasapainojakauma.
\end{lause}

\begin{proof}
	Yhtälön \ref{siirt-tiheys} mukaan $\int_S T(y,x)dx = 1$, joten
	\begin{equation}
		\int_S \pi(x) T(x,y) dx = \int_S \pi(y) T(y,x) dx = \pi(y)\int_S T(y,x) dx = \pi(y)
	\end{equation}
\end{proof}

\begin{esim}
	Pohditaan lyhyttä esimerkkiä, jossa tilajoukko on $S=\{"sataa", "paistaa"\}$. Määritellään siirtymätodennäköisyydet siirtymämatriisilla
	\begin{displaymath}
		\pr ^{(1)}=  
		\begin{pmatrix}
			0.7 & 0.3 \\
			0.2 & 0.8
		\end{pmatrix}
	\end{displaymath}
	Tämä voidaan visualisoida kuvan \ref{mc-esimerkki1} mukaisesti.
	\begin{figure}[h!]
	\label{mc-esimerkki1}
	\begin{center}
		\begin{tikzpicture}[->,>=stealth',shorten >=1pt,auto,node distance=3cm,semithick]

		\node[state](s){Paistaa};
    	\node[state, right of=s](r){Sataa};
		\draw[every loop]
            (s) edge[bend left] node {0.2} (r)
            (r) edge[bend left] node {0.3} (s)
            (r) edge[loop right] node {0.7} (r)
            (s) edge[loop left] node {0.8} (s);
	\end{tikzpicture}
	\caption[Yksinkertainen esimerrki Markovin ketjusta]{Esimerkki \ref{mc-esimerkki1}}
	\end{center}
	\end{figure}
	Ketju on äärellinen, joten sillä on tasapainojakauma. Yhtälö \ref{stationary-dist2} implikoi, että jakauma $\pi$ on siirtymämatriisin $\pr$ vasen ominaisvektori ($\pi^T\pr=\lambda\pi^T$, jossa $\lambda=1$). Tämä voidaan ratkaista numeerisesti, ja ratkaisu on $\pi^T=(0.4, 0.6)$. Helposti nyt nähdään, että \ref{stationary-dist2} pätee.
\end{esim}

Seuraavaksi esitellään lyhyeksi \textit{syklinen siirtymäydin}, ja osoitetaan tulos, jota tarvitsemme Gibbsin otanta-algoritmin tasapainojakauman todistuksen yhteydessä.

%%%%%%%%%%%%%%%%%%%%%%%%%%%%%%%%%%%%%%%%%%%%%%%%%%%%%%%%%%
%%%%%%%%%%%%%%%%%%%%%%%%%%%%%%%%%%%%%%%%%%%%%%%%%%%%%%%%%%
%%%%%%%%%%%%%%%%%%%%%%%%%%%%%%%%%%%%%%%%%%%%%%%%%%%%%%%%%%
%%%%%%%%%%%%%%%%%%%%%%%%%%%%%%%%%%%%%%%%%%%%%%%%%%%%%%%%%%

\begin{maar}
	Markovin ketjuissa voimme myös tietyin ehdoin yhdistää siirtymätiheyksiä. Tällöin puhutaan syklisistä siirtymätiheyksistä. Voidaan esimerksiksi määritellä 	siirtymätiheys
	\begin{equation}
		T_1...T_d
	\end{equation}
\end{maar}

\begin{lause}\label{cyclic-kernel}
	Jos $\pi$ on tasapainojakauma kaikille siirtymätiheyksille $T_1,...,T_d$, niin se on tasapainojakauma siirtymätiheydelle $T = T_1...T_d$.
\end{lause}

\begin{proof}
	\begin{equation*}
		\pi T = \pi T_1...T_d = \pi T_2...T_d = ... = \pi T_d = \pi
	\end{equation*}
\end{proof}

Käydään vielä nopeasti läpi muutama ominaisuus, joita tarvitaan kun haluamme osoittaa kappaleessa 3 algoritmiemme toimivan.

\begin{maar}
	Markovin ketju $(X_n)$, siirtymäytimellä $T(x,y)$ on \textit{pelkistymätön} jos kaikilla $A \in \mathcal{T}$, joilla $\pr(A)>0$, on olemassa sellainen $n$, että $T^n(x, A)>0$ kaikilla $x\in S$
\end{maar} 

Käytännössä pelkistymättömyys tarkoittaa siis sitä, että jokaisesta tila-avaruuden kolkasta on mahdollista päästä jokaiseen muuhun pisteeseen avaruutta, eli ketju ei voi jäädä jumiin johonkin alueelle.

\begin{maar}
	Markovin ketju $(X_n)$ on \textit{palautuva} jos
	\begin{enumerate}[(i)]
		\item ketju $(X_n)$ on pelkistymätön ja
		\item kaikilla $A\in \mathcal{T}$, joilla $\pr(A)>0$, $\mathbb{E}_x [\eta_A]=\infty$ kaikilla $x\in A$
	\end{enumerate}
	Missä $\eta_A$ on käyntien määrä joukossa $A$.
\end{maar}

Palautuvalla ketju tarkoittaa sitä, että ketju palaa alueelle, jossa se on jo käynyt. Palautumista vahvempi ominaisuus on \textit{Harris}-palautuvuus.

\begin{maar}
	Joukko $A$ on Harris palautuva, jos $\pr_x(\eta_A =\infty) = 1$ kaikilla $x\in A$. Markovin ketju $(X_n)$ on Harris palautuva jos $(X_n)$ on pelkistymätön, ja jokainen joukko $A$, jolla $\pr(A)>0$, on Harris palautuva.
\end{maar}

\begin{maar}
	Markovin ketju $(X_n)$ on perioidinen, jos on olemassa erilliset osajoukot $A_1, A_2,...,A_d \subset S, d > 1$, että 
	\begin{equation}
		T(x, A_{i+1}) = 1 , \: \forall \in A_i, \: i=1,...,d-1
	\end{equation}
	ja 
	\begin{equation}
		T(x,A_1)=1, \: i = d
	\end{equation}
	Jos ketju ei ole perioidinen, se on \textit{aperiodinen}.
\end{maar}

Määritellään sitten viekä ergodisuus, joka on tärkeä ominaisuus Markovin ketjun tasapainojakauman olemassaolon kannalta.

\begin{maar}
	Markovin ketju $(X_n)$ on ergodinen, jos se on pelkistymätön, aperiodinen ja Harris palautuva.
\end{maar}

Ergodisuus on MCMC-menetelmien kannalta tärkeä ominaisuus, sillä se takaa Markovin ketjun $(X_n)$ konvergoitumisen uniikkiin tasapainojakaumaansa mistä tahansa tila-avaruuden $S$ pisteestä. Tämän osoittaminen on melko hankalaa ja ylittää reilusti tämän tutkielman laajuuden, joten jätämme sen tekemättä.



