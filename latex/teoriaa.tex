


\chapter{Yleisi� tuloksia}\label{teor}

\section{Perusm��ritelmi�}
M��ritell��n ensiksi todenn�k�isyys.
\begin{maar}
	\textit{$\sigma$-algebra.} Olkoot $\Omega$ mielivaltainen ep�tyhj� joukko. Sigma-algebra perusjoukolla $\Omega$ on sen osajoukkojen joukkoperhe $\mathcal{F}$, joka toteuttaa ehdot:
	
	\begin{enumerate}
		\item $\emptyset\in\mathcal{F}$
		\item jos $A\in\mathcal{F},\ niin\ A^c \in\mathcal{F}$
		\item jos jos $A_k\in\mathcal{F},\ kaikilla\ k\in K$, miss� $K$ on numeroituva joukko, niin $\bigcup_{k\in K} A_k \in \mathcal{F}$
	\end{enumerate}
\end{maar}

\begin{maar}
	Kuvaus \pr\ liitt�� kuhunkin tapahtumaan $A$ \textit{todenn�k�isyyden}, joka on luku suljetulla v�lill� [0,1] ja sille p�tee:
	\begin{enumerate}
		\item $\pr(\Omega)=1$
		\item Jos $A$ on tapahtuma, niin sen komplementtitapahtuman $A^c$ todenn�k�isyys on $\pr(A^c)= 1 - \pr(A)$
		\item Jos $(A_k)_{k\in\N}$ ovat erillisi� tapahtumia, niin 
		\begin{displaymath}
			\pr(\bigcup_{k\in\N} A_k) = \sum_{k\in\N}\pr(A_k)
		\end{displaymath}
	\end{enumerate}
\end{maar}

\begin{maar}
	Kolmikkoa $(\Omega, \mathcal{F}, \pr )$ kutsutaan todenn�k�isyysavaruudeksi.
\end{maar}

\section{Markovin ketjut}