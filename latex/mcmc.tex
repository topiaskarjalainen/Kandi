\chapter{Metropolis--Hastings algoritmi}

\textit{Metropolis--Hastings} algoritmi on kehittelij�idenss� Nicholas Metropolisksen (1915-1999) ja \textit{Wilfred Keith Hastings}:n (1930-2016) mukaan nimetty MCMC-menetelm�, jolla voidaan simuloida Bayesil�isess� analyysissa k�ytett�vi� posteriori jakaumia my�s silloin kun tiheys on mahdotonta m��ritt�� analyyttisesti.

\begin{maar}\label{mh-maar}
	Metropolis--Hastings algoritmi on seuraavanlainen
	\begin{enumerate}
		\item Valitaan aloitus tila $\theta_0$ ja asetetaan $t=0$
		\item Generoidaan kandidaatti tila $\theta'$ satunnaisesti jakaumasta $J_t(\theta'|\theta_{t-1})$
		\item Lasketaan tiheyksien tai todenn�k�isyyksien suhde
		\begin{displaymath}
			r = \frac{p(\theta'|y)/J_t(\theta'|\theta^{t-1})}{p(\theta^{t-1}|y)/J_t(\theta^{t-1}|\theta')}
		\end{displaymath}
		\item Asetetaan
		\begin{displaymath}
			\theta_t= 
			\begin{cases}
				\theta', \text{todenn�k�isyydell�} \hspace{0.3cm} \min(r,1) \\
				\theta_{t-1}, \text{muuten}
			\end{cases}
		\end{displaymath}
	\end{enumerate}
	Jossa $J_t(\theta'|\theta^{t-1})$ on ns. ehdotusjakauma (eng. proposal distribution).
\end{maar}

\begin{lause}
	M��ritelm�n \ref{mh-maar} algoritmi tuottaa Markovin ketjun jolla on uniikki tasapainojakauma, ja jonka tasapainojakauma on haluttu jakauma $p(\theta|y)$, jossa $y$ on data. 
\end{lause}

\begin{proof}
	Ohitamme todistuksen, ett� kyseess� Markovin ketju jolla yksi tasapainojakauma, mutta todistamme toisen osan, eli ett� tasapainojakauma on haluttu $p(\theta|y)$ eli posteriori jakauma. 
\end{proof}