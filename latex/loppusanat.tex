\chapter{Loppusanat}

Olemme nyt tarkastelleet perus ajatuksia MCMC-menetelmistä. Kävimme läpi hieman teoriaa, esittelimme algoritmit ja kokeilimme sellaista käytännön tilanteessa. Kuitenkin olemme vain raapaisseet pintaa. Käyttämämme algoritmit ovat monessa mielessä erittäin rajoittuneita. Gibbsin otanta-algortimissa meidän tulee tuntea marginaalijakaumat, ja Metropolis--Hastings voi olla välillä hidas konvergoitumaan. 

Tärkeitä laajennuksia, jotka korjaavat edellisessäkappaleessa mainitsemiani onglemia, on myöhemmin kehitetyt tehokkaammat MCMC-metodit. Tärkeimpänä \textit{Hamiltonian monte carlo} eli HMC. Se pyrkii vähentämään tarvittavien otosten määrää, joka tarvitaan tarpeeksi tarkan posteriori estimaatin saamiseksi, sillä menetelmä vähentää autokorrelaatiota tilojen välillä.

Ohitimme myös useita asioita liittyen Markovin ketjuihin. Markovin ketjujen matematiikka ei ole  kovin vaikeaa käytännössä, mutta monet todistukset ovat työläitä ja vaativat matematiikkaa, joka ylittää tämän tutkielman laajuuden.

Kuitenkin MCMC-menetelmät ovat mullistaneet maailmaa muunmuassa koneoppimis menetelmien massakaupallistumisen myötä. Ne ovat myös tehneet tieteestä luotettavampaa mahdollistamalla Bayesiläisen tilastotieteen uuden nousun. Varsinkin replikaatiokriisin jälkeisessä maailmassa se on arvokas kontribuutio.

Paljon on kuitenkin vielä tutkittavaa. Suurista viimeaikaisista loikista huolimatta, MCMC-menetelmtä eivät voita vieläkään nopeudessa perinteisiä tilastollisia menetelmiä, joten uusia ja nopeampia algoritmeja olisi vielä kehiteltävä ja helppokäyttöisyyttä parannettava, jotta MCMC-menetelmät saavuttavat täyden potentiaalinsa.