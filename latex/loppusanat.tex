\chapter{Loppusanat}

Olemme nyt tarkastelleet perus ajatuksia MCMC-menetelmistä. Kävimme läpi hieman teoriaa, esittelimme algoritmit ja kokeilimme sellaista käytännön tilanteessa. Kuitenkin olemme vain raapaisseet pintaa. Käyttämämme algoritmit ovat monessa mielessä erittäin rajoittuneita. Gibbsin otanta-algortimissa meidän tulee tuntea marginaalijakaumat analyyttisesti, ja Metropolis--Hastings voi olla välillä hidas konvergoitumaan. 

Tärkeitä laajennuksia, jotka korjaavat edellisessäkappaleessa mainitsemiani onglemia, on myöhemmin kehitetyt tehokkaammat MCMC-metodit. Tärkeimpänä \textit{Hamiltonian monte carlo} eli HMC. Se pyrkii vähentämään tarvittavien otosten määrää, joka tarvitaan tarpeeksi tarkan posteriori estimaatin saamiseksi, sillä menetelmä vähentää autokorrelaatiota tilojen välillä.

