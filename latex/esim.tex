\chapter{Laajempi esimerkki}

Tarkastellaan vielä lopuksi laajempaa esimerkkiä. Tarkastellaan normaalia lineaarista regressiomallia. Käytetään \texttt{R}:stä löytyvää \texttt{MASS::Boston} dataa, jossa on dataa Bostinin lähiöiden talojen mediaanihinnoista. Regressoidaan datasetin muuttuja \texttt{medv} muuttujille \texttt{dis} ja \texttt{crim}, eli regressoidaan \textit{asunnon mediaanihinta} muuttujille \textit{etäisyys yrityskyliin} ja \textit{rikollisuuden määrä}. Sovitetaan tämä malli Gibbsin otanta-algoritmin avulla. Sitä varten meidän on ensiksi määriteltävä marginaaliset ehdolliset jakaumat parametreille.

Määritellään ensiksi malli. Merkitään $\tau = 1/\sigma^2$, $\theta = (\beta_0,\beta_1,\beta_2,\tau)$
\begin{equation}
\begin{split}
	y_i|\beta_0,\beta_1,\beta_2,\tau &\sim N(\beta_0+\beta_1 x_i+\beta_2x_i, 1/\tau) \\
	\beta_0|\mu_0,\tau_0 &\sim N(\mu_0, 1/\tau_0) \\
	\beta_1|\mu_1,\tau_1 &\sim N(\mu_1, 1/\tau_1) \\
	\beta_2|\mu_2,\tau_2 &\sim N(\mu_2, 1/\tau_2) \\
	\tau|\alpha, \gamma &\sim Gamma(\alpha, \gamma)
\end{split}	
\end{equation}
Muodostetaan uskottavuusfunktio
\begin{equation}
	L(y|\theta) = \prod_{i=1}^{N} N(\beta_0+\beta_1 x_i+\beta_2x_i, 1/\tau)
\end{equation}
Ja posterioiri on siten 
\begin{equation}
	p(\theta|y) \propto p(\beta_0)p(\beta_1)p(\beta_2)p(\tau)\prod_{i=1}^{N} N(\beta_0+\beta_1 x_i+\beta_2x_i, 1/\tau)
\end{equation}
Tästä saadaan sitten johdettua marginaaliset jakaumat otanta-algoritmia varten
\begin{equation}
\begin{split}
	\beta_0 | \beta_1,\beta_2,\tau_0,\tau,\mu_0,x,y &\sim 
	N\pqty{\frac{\tau_0\mu_0+\tau\sum_{i=1}^{N}(y_i-\beta_1x_{i1}-\beta_2x_{i2})}{\tau_0+\tau N}, \frac{1}{\tau_0+\tau N}} \\
	\beta_1 | \beta_0,\beta_2,\tau_1,\tau,\mu_1,x,y &\sim 
	N\pqty{\frac{\tau_1\mu_1+\tau\sum_{i=1}^{N}(y_i-\beta_0-\beta_2x_{i2})x_{i1}}{\tau_1+\tau\sum_{i=1}^{N}x_{i1}^2}, \frac{1}{\tau_1+\tau\sum_{i=1}^{N}x_{i1}^2}}\\
	\beta_2 | \beta_0,\beta_1,\tau_2,\tau,\mu_2,x,y &\sim 
	N\pqty{\frac{\tau_2\mu_2+\tau\sum_{i=1}^{N}(y_i-\beta_0-\beta_1x_{i1})x_{i2}}{\tau_2+\tau\sum_{i=1}^{N}x_{i2}^2}, \frac{1}{\tau_2+\tau\sum_{i=1}^{N}x_{i2}^2}}\\
	\tau | \beta_0,\beta_1,\beta_2,\alpha,\gamma,x,y &\sim 
	Gamma\pqty{\alpha+\frac{N}{2},\gamma + \frac{N}{2}\sum_{i=1}^{N}(y_i-\beta_0-\beta_1x_{i1}-\beta_2x_{i2})^2}
\end{split}
\end{equation}
Asetetaan aloitusarvot