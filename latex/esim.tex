\chapter{Laajempi esimerkki}

Tarkastellaan vielä lopuksi laajempaa esimerkkiä. Tarkastellaan normaalia lineaarista regressiomallia. Käytetään \texttt{R}:stä löytyvää \texttt{airquality} dataa, jossa on dataa New Yorkin ilmanlaadusta. Regressoidaan datasetin muuttuja \texttt{Ozone} muuttujille \texttt{Solar.R} ja \texttt{Wind}. Sovitetaan tämä malli Gibbsin otanta-algoritmin avulla. Sitä varten meidän on ensiksi määriteltävä marginaaliset ehdolliset jakaumat parametreille.

Määritellään ensiksi malli. Merkitään $\tau = 1/\sigma^2$, $\theta = (\beta_0,\beta_1,\beta_2,\tau)$
\begin{equation}
\begin{split}
	y_i|\beta_0,\beta_1,\beta_2,\tau &\sim N(\beta_0+\beta_1 x_i+\beta_2x_i, 1/\tau) \\
	\beta_0|\mu_0,\tau_0 &\sim N(\mu_0, 1/\tau_0) \\
	\beta_1|\mu_1,\tau_1 &\sim N(\mu_1, 1/\tau_1) \\
	\beta_2|\mu_2,\tau_2 &\sim N(\mu_2, 1/\tau_2) \\
	\tau|\alpha, \gamma &\sim Gamma(\alpha, \gamma)
\end{split}	
\end{equation}
Muodostetaan uskottavuusfunktio
\begin{equation}
	L(y|\theta) = \prod_{i=1}^{N} N(\beta_0+\beta_1 x_i+\beta_2x_i, 1/\tau)
\end{equation}
Ja posterioiri on siten 
\begin{equation}
	p(\theta|y) \propto p(\beta_0)p(\beta_1)p(\beta_2)p(\tau)\prod_{i=1}^{N} N(\beta_0+\beta_1 x_i+\beta_2x_i, 1/\tau)
\end{equation}
Tästä saadaan sitten johdettua marginaaliset jakaumat otanta-algoritmia varten
\begin{equation}\label{marginaalit}
\begin{split}
	\beta_0 | \beta_1,\beta_2,\tau_0,\tau,\mu_0,x,y &\sim 
	N\pqty{\frac{\tau_0\mu_0+\tau\sum_{i=1}^{N}(y_i-\beta_1x_{i1}-\beta_2x_{i2})}{\tau_0+\tau N}, \frac{1}{\tau_0+\tau N}} \\
	\beta_1 | \beta_0,\beta_2,\tau_1,\tau,\mu_1,x,y &\sim 
	N\pqty{\frac{\tau_1\mu_1+\tau\sum_{i=1}^{N}(y_i-\beta_0-\beta_2x_{i2})x_{i1}}{\tau_1+\tau\sum_{i=1}^{N}x_{i1}^2}, \frac{1}{\tau_1+\tau\sum_{i=1}^{N}x_{i1}^2}}\\
	\beta_2 | \beta_0,\beta_1,\tau_2,\tau,\mu_2,x,y &\sim 
	N\pqty{\frac{\tau_2\mu_2+\tau\sum_{i=1}^{N}(y_i-\beta_0-\beta_1x_{i1})x_{i2}}{\tau_2+\tau\sum_{i=1}^{N}x_{i2}^2}, \frac{1}{\tau_2+\tau\sum_{i=1}^{N}x_{i2}^2}}\\
	\tau | \beta_0,\beta_1,\beta_2,\alpha,\gamma,x,y &\sim 
	Gamma\pqty{\alpha+\frac{N}{2},\gamma + \frac{N}{2}\sum_{i=1}^{N}(y_i-\beta_0-\beta_1x_{i1}-\beta_2x_{i2})^2}
\end{split}
\end{equation}
Asetetaan hyperparametrit $\mu_0 = 80,\mu_1 = 0, \mu_2 = -5, \tau_0 = 1/50, \tau_1 = 1/50, \tau2 = 1/50, \alpha	= 5, \gamma = 0.01$. Parametrit on valittu sen mukaan, että ne asettavat suurimman tiheyden lähelle arvioitua sijaintia ja toisaalta eivät ole kovin informatiivisia vaan leveitä. Simuloidaan nyt tästä Gibbsin otanta-algoritmilla yhtälöitä \ref{marginaalit} käyttäen 8 ketjua, kunkin pituus 20 000. Burnin-periodi olkoot 10 000. Yhteensä siis meillä on 80 000 otosta.

% latex table generated in R 3.6.1 by xtable 1.8-4 package
% Tue Apr  7 16:14:05 2020
\begin{table}[ht]
\centering
\label{results}
\begin{tabular}{rrrrrrrr}
  \hline
 $\theta$ & Mean & SD & 2.5\% & 25\% & 50\% & 75\% & 97.5\% \\ 
  \hline
  $\beta_0$ & 75.95307 & 2.11884 & 72.06682 & 74.62170 & 75.95679 & 77.29428 & 79.82864 \\ 
  $\beta_1$ & 0.06646 & 0.01001 & 0.04698 & 0.05973 & 0.06645 & 0.07320 & 0.08612 \\ 
  $\beta_2$ & -5.30405 & 0.20235 & -5.68918 & -5.43539 & -5.30480 & -5.17319 & -4.92167 \\ 
  $\tau$ & 0.00173 & 0.00022 & 0.00132 & 0.00158 & 0.00172 & 0.00188 & 0.00219 \\ 
   \hline
\end{tabular}
\caption{Tulokset regressiosta}
\end{table}
\begin{table}[ht]\label{diagnostics}
\centering
\begin{tabular}{rrr}
  \hline
 $\theta$ & $\hat{R}$ & $\hat{n}_{eff}$ \\ 
  \hline
  $\beta_0$ & 1 & 80008 \\ 
  $\beta_1$ & 1 & 80008 \\ 
  $\beta_2$ & 1 & 80008  \\ 
  $\tau$ & 1 & 80008  \\ 
   \hline
\end{tabular}
\caption{Diagnostiikka}
\end{table}

Taulukoissa 4.1 ja 4.2 nähdään tulokset ja MCMC-diagnostiikat, joista puhuttiin aiemmin. Nähdään, että ketjut näyttäisivät konvergoituneen, mikä ei sinänsä ihmetytä lainkaan, sillä simulaatio määrämme on valtava.
\begin{figure}[h!]
	\includegraphics[width=\textwidth]{gibbsexample}
	\caption[Regressio]{\textit{Esimerkin regression tulos, vasemmalla Solar.R -akselilla ja oikealla Wind. Punainen viiva on posteriori keskiarvolla piiretty regressio viiva ja vihreä on vertailun vuoksi PNS mentelmällä sovitettu suora. Siniset ovat sovitteita parametrien eri ehdotus arvoilla.}}
	\label{kuva1}
\end{figure}

\begin{figure}[h!]
	\includegraphics[width=\textwidth]{gibbs2}
	\caption[]{\textit{Esimerkin ketjut ja parametrien posteriori tiheydet}}
	\label{kuva1}
\end{figure}



