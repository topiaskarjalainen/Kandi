\section{Konvergenssi}

Kappaleissa \ref{gibbs} ja \ref{Metropolis--Hastings algoritmi} esiteltyjen algoritmien kohdalla voi her�t� kysymys, ett� mik� on riitt�v� m��r� iteraatioita, jotta Markovin ketju on saavuttanut tasapainojakaumansa ja otanta on riitt�v�n hyv� aproksimaatio posteriorijaukaumasta. Esimerkiksi kun katsotaan kuvan \ref{kuva1} vasemman puolen kuvan punaista polkua, niin voimme sanoa, ett� se ei ole viel� saavuttanut tasapainojakaumaa sill� tunnemme melko hyvin halutun jakauman, mutta yleens� emme v�ltt�m�tt� osaa sanoa t�t� suoraan.
